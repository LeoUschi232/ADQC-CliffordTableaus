\begin{frame}{Pauli Group}{Definitions}
    \(\mathcal{P}_n\)
    is defined as the group of \(n\)-qubit Pauli operators. \\
    It consists of all tensor products of \(n\) Pauli matrices, with a phase factor \(\pm 1\) or \(\pm i\).
    \begin{gather*}
        \onslide<2->{
            \mathcal{P}_1=\left\{
            \pm I,\pm iI,
            \pm X,\pm iX,
            \pm Y,\pm iY,
            \pm Z,\pm iZ
            \right\} \\
        }
        \onslide<3->{
            \mathcal{P}_n
            =\left\{
            \left.
            i^m\bigotimes_{j=1}^n\sigma_{k_j}
            \right|
            m,k_j\in\left\{0,1,2,3\right\},
            \sigma_0=I,
            \sigma_1=X,
            \sigma_2=Y,
            \sigma_3=Z
            \right\}
        }
    \end{gather*}
    \onslide<4->{
        \newline
        Size of a Pauli Group:
        \(\left|\mathcal{P}_n\right|=4^{n+1}\)
    }

    \vspace*{5mm}

    \fullfootcite{02_ImprovedSimulationOfStabilizerCircuits} \\
    \fullfootcite{01_QuantumComputationAndQuantumInformation}
\end{frame}

\begin{frame}{Pauli Group}{Operations}
    Given two Pauli operators
    \(P=i^{m_P}\bigotimes_{j=1}^{n}P_j\)
    and
    \(Q=i^{m_Q}\bigotimes_{j=1}^{n}Q_j\),
    their product, as necessitated by Group Definition, is:
    \[
        P\cdot Q=i^{m_P+m_Q}\bigotimes_{j=1}^{n}P_jQ_j
    \]
    \onslide<2->{
        \newline
        \(P\) commutes with \(Q\) if the number of indices \(j\) such that \(P_j\) anti-commutes with \(Q_j\) is even.
    }

    \vspace*{25mm}

    \fullfootcite{02_ImprovedSimulationOfStabilizerCircuits}
\end{frame}
