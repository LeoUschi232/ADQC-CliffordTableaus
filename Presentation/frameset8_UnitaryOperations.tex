\begin{frame}{Unitary Operations}{Main Revelation}
    Suppose \(U\) is a unitary operator, \(\ket{\psi}\in V_S\) and \(g\in S\).
    \onslide<2->{
        \[
            U\ket{\psi}
            =Ug\ket{\psi}
            =UgI\ket{\psi}
            =UgU^{\dagger}U\ket{\psi}
            =\left(UgU^{\dagger}\right)U\ket{\psi}
        \]
    }
    \onslide<3->{
        \(\Longrightarrow\)
        State \(U\ket{\psi}\) is stabilized by \(UgU^{\dagger}\).
    }

    \vspace*{2mm}

    \onslide<4->{
        \(\Longrightarrow\)
        If we can describe a state by its stabilizers, we can easily compute the stabilizers of the state
        that emerges from the previous state under a unitary operation.
    }

    \vspace*{25mm}

    \fullfootcite{01_QuantumComputationAndQuantumInformation}
\end{frame}


\begin{frame}{Unitary Operations}{Advantages for Computation}
    For certain special unitary operations \(U\) this transformation of the generators
    takes on a particularly appealing form.
    \onslide<2->{
        \[
            HXH^{\dagger}=Z
            \qquad
            HYH^{\dagger}=-Y
            \qquad
            HZH^{\dagger}=X
        \]
    }

    \vspace*{2mm}

    \onslide<3->{
        \textbf{Example:} \\
        (Unkown) State \(\ket{\psi}\) stabilized by \(X\). \\
    }
    \onslide<4->{
        \(\qquad\longrightarrow\)
        Apply Hadamard gate \(H\) to \(\ket{\psi}\). \\
    }
    \onslide<5->{
        \(\qquad\qquad\Longrightarrow\)
        Resulting (Unkown) state \(\ket{\psi'}\) stabilized by \(Z\).
    }

    \vspace*{25mm}

    \fullfootcite{01_QuantumComputationAndQuantumInformation}
\end{frame}

\begin{frame}{Unitary Operations}{Transformation Properties}
    \[
        \renewcommand{\arraystretch}{1.0}
        \begin{array}{|c|c|c|}
            \hline
            \textbf{Operation}    & \textbf{Input} & \textbf{Output} \\
            \hline
            \text{controlled-NOT} & X_1            & X_1 X_2         \\
            & X_2            & X_2             \\
            & Z_1            & Z_1             \\
            & Z_2            & Z_1 Z_2         \\
            \hline
            H                     & X              & Z               \\
            & Z              & X               \\
            \hline
            S                     & X              & Y               \\
            & Z              & Z               \\
            \hline
        \end{array}
        \quad
        \begin{array}{|c|c|c|}
            \hline
            \textbf{Operation} & \textbf{Input} & \textbf{Output} \\
            \hline
            X                  & X              & X               \\
            & Z              & -Z              \\
            \hline
            Y                  & X              & -X              \\
            & Z              & -Z              \\
            \hline
            Z                  & X              & -X              \\
            & Z              & Z               \\
            \hline
        \end{array}
    \]

    \vspace*{10mm}

    \fullfootcite{01_QuantumComputationAndQuantumInformation}
\end{frame}