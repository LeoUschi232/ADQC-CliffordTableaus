\begin{frame}{Stabilizer Groups}{Definitions}
    \begin{itemize}
        \setlength{\itemsep}{1.25\baselineskip}
        \item
        Element \(g\in \mathcal{P}_n\) \textbf{stabilizes} \(\ket{\psi}\) iff \(g\ket{\psi}=\ket{\psi}\). \\
        \(\ket{\psi}\) is eigenstate of \(g\) with eigenvalue \(+1\).
        \item
        \(S\widehat{=}\)Subgroup of the Pauli Group \(\mathcal{P}_n\): \(S\subseteq\mathcal{P}_n\).
        \item
        \(V_S\widehat{=}\)Set of \(n\)-qubit states stabilized by \(S\):
    \end{itemize}
    \vspace*{4mm}
    \[
        V_S=\left\{
        \ket{\psi}
        \mid
        S\subseteq \mathcal{P}_n,
        \forall g\in S\holds
        g\ket{\psi}=\ket{\psi}
        \right\}
    \]
    \vspace*{15mm}

    \fullfootcite{01_QuantumComputationAndQuantumInformation}
\end{frame}


\begin{frame}{Stabilizer Groups}{Properties}
    Not just any subgroup \(S\) of the Pauli group can be used as the stabilizer
    for a non-trivial vector space \(V_S\). \\
    \vspace*{2mm}
    \textbf{Example:}
    \(S=\{\pm I,\pm X\}\)
    \[
        (-I)\in S
        \aand
        (-I)\ket{\psi}=-\ket{\psi}
        \means
        \ket{\psi}=\Vec{0}
        \means
        V_S=\begin{Bmatrix}
                \Vec{0}
        \end{Bmatrix}\,\text{(trivial)}
    \]

    \vspace*{3mm}
    Conditions for \(S\) such that \(V_S\) not trivial:
    \vspace*{2mm}
    \begin{itemize}
        \setlength{\itemsep}{0.75\baselineskip}
        \item \textbf{Commutativity:}
        \(\forall g_1,g_2\in S\holds g_1g_2=g_2g_1\)
        \item \textbf{Strict Identity:}
        \(-I\not\in S,\,iI\not\in S,\,-iI\not\in S\)
    \end{enumerate}
    \vspace*{7mm}

    \fullfootcite{01_QuantumComputationAndQuantumInformation}
\end{frame}