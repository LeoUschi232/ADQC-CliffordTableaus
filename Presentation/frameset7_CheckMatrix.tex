\begin{frame}{Check Matrix}{Structure}
    Suppose \(S=\left\langle g_i\mid i\in\natur,1\leq i\leq l\right\rangle\). \\

    \vspace*{2mm}
    \(\textbf{Check Matrix }H_S\widehat{=}\)Extremely useful way of presenting the generators \\

    \vspace*{2mm}
    \(H_S\) is an \(l\times 2n\) binary matrix whose rows correspond to the generators \(g_1\) through \(g_l\). \\
    \[
        \text{Example:}\qquad
        l\left\{\begin{matrix}
                    \, \\ \, \\ \, \\ \, \\ \, \\ \,
        \end{matrix}\right.
        \left[\begin{matrix}
                  \, \\ \, \\ \, \\ \, \\ \, \\ \,
        \end{matrix}\right.
        \underbrace{
            \begin{matrix}
                0 & 0 & 0 & 1 & 1 & 1 & 1 \\
                0 & 1 & 1 & 0 & 0 & 1 & 1 \\
                1 & 0 & 1 & 0 & 1 & 0 & 1 \\
                0 & 0 & 0 & 0 & 0 & 0 & 0 \\
                0 & 0 & 0 & 0 & 0 & 0 & 0 \\
                0 & 0 & 0 & 0 & 0 & 0 & 0
            \end{matrix}
        }_{n}
        \left|\begin{matrix}
                  \, \\ \, \\ \, \\ \, \\ \, \\ \,
        \end{matrix}\right.
        \underbrace{
            \begin{matrix}
                0 & 0 & 0 & 0 & 0 & 0 & 0 \\
                0 & 0 & 0 & 0 & 0 & 0 & 0 \\
                0 & 0 & 0 & 0 & 0 & 0 & 0 \\
                0 & 0 & 0 & 1 & 1 & 1 & 1 \\
                0 & 1 & 1 & 0 & 0 & 1 & 1 \\
                1 & 0 & 1 & 0 & 1 & 0 & 1
            \end{matrix}
        }_{n}
        \left]\begin{matrix}
                  \, \\ \, \\ \, \\ \, \\ \, \\ \,
        \end{matrix}\right.
    \]

    \vspace*{2mm}

    \fullfootcite{01_QuantumComputationAndQuantumInformation}
\end{frame}


\begin{frame}{Check Matrix}{Interpretation}
    \begin{itemize}
        \setlength{\itemsep}{0.25\baselineskip}
        \item
        Row \(i\) corresponds to generator \(g_i\in S\).
        \item
        Left \(l\times n\) submatrix contains 1s to indicate which generators contain \(X\)s.
        \item
        Right \(l\times n\) submatrix contains 1s to indicate which generators contain \(Z\)s.
        \item
        Presence of 1 in both submatrices indicates \(Y\) in that generator.
    \end{itemize}

    \vspace*{2mm}
    More explicitly, with \(h_{i,j}\) denoting the element of \(H_S\) at row \(i\) and column \(j\):
    \vspace*{2mm}
    \begin{itemize}
        \setlength{\itemsep}{0.25\baselineskip}
        \item
        If \(g_i\) contains \(I\) on the \(j^{\text{th}}\) qubit
        \(\Longrightarrow\)
        \(h_{i,j}=0\) and \(h_{i,n+j}=0\).
        \item
        If \(g_i\) contains \(X\) on the \(j^{\text{th}}\) qubit
        \(\Longrightarrow\)
        \(h_{i,j}=1\) and \(h_{i,n+j}=0\).
        \item
        If \(g_i\) contains \(Z\) on the \(j^{\text{th}}\) qubit
        \(\Longrightarrow\)
        \(h_{i,j}=0\) and \(h_{i,n+j}=1\).
        \item
        If \(g_i\) contains \(Y\) on the \(j^{\text{th}}\) qubit
        \(\Longrightarrow\)
        \(h_{i,j}=1\) and \(h_{i,n+j}=1\).
    \end{itemize}

    \vspace*{4mm}

    \fullfootcite{01_QuantumComputationAndQuantumInformation}
\end{frame}

\begin{frame}{Check Matrix}{Example Steane Code}
    For Readability tensor product operator signs are left out.
    \(\sigma_i\sigma_j\) corresponds to \(\sigma_i\otimes\sigma_j\). \\
    \[
        \left[\begin{matrix}
                  \, \\ \, \\ \, \\ \, \\ \, \\ \,
        \end{matrix}\right.
        \begin{matrix}
            0 & 0 & 0 & 1 & 1 & 1 & 1 \\
            0 & 1 & 1 & 0 & 0 & 1 & 1 \\
            1 & 0 & 1 & 0 & 1 & 0 & 1 \\
            0 & 0 & 0 & 0 & 0 & 0 & 0 \\
            0 & 0 & 0 & 0 & 0 & 0 & 0 \\
            0 & 0 & 0 & 0 & 0 & 0 & 0
        \end{matrix}
        \left|\begin{matrix}
                  \, \\ \, \\ \, \\ \, \\ \, \\ \,
        \end{matrix}\right.
        \begin{matrix}
            0 & 0 & 0 & 0 & 0 & 0 & 0 \\
            0 & 0 & 0 & 0 & 0 & 0 & 0 \\
            0 & 0 & 0 & 0 & 0 & 0 & 0 \\
            0 & 0 & 0 & 1 & 1 & 1 & 1 \\
            0 & 1 & 1 & 0 & 0 & 1 & 1 \\
            1 & 0 & 1 & 0 & 1 & 0 & 1
        \end{matrix}
        \left]\begin{matrix}
                  \, \\ \, \\ \, \\ \, \\ \, \\ \,
        \end{matrix}\right.
        \widehat{=}
        \,\,\,
        {
            \renewcommand{\arraystretch}{1.25}
            \begin{array}{|c|c|}
                \hline
                \text{Generator} & \text{Operator} \\
                \hline
                g_1              & I\,I\,I\,XXXX   \\
                g_2              & I\,XXI\,I\,XX   \\
                g_3              & XI\,XI\,XI\,X   \\
                g_4              & I\,I\,I\,ZZZZ   \\
                g_5              & I\,ZZI\,I\,ZZ   \\
                g_6              & ZI\,ZI\,ZI\,Z   \\
                \hline
            \end{array}
        }
    \]

    \vspace*{4mm}

    \fullfootcite{01_QuantumComputationAndQuantumInformation}
\end{frame}