\documentclass[english,aspectratio=169]{tumbeamer}


% Packages with no parameters
\usepackage{booktabs}
\usepackage{amsmath}
\usepackage{amssymb}
\usepackage{braket}
\usepackage{dsfont}
\usepackage{etoolbox}
\usepackage{float}
\usepackage{graphicx}
\usepackage{tikz}
\usepackage{setspace}

% Packages with parameters
\usepackage[left=15mm, right=15mm, top=7mm, bottom=7mm]{geometry}

% Personlized math operators
\DeclareMathOperator{\tacticalAnd}{\qquad\land\qquad}
\DeclareMathOperator{\means}{\quad\Longrightarrow\quad}
\DeclareMathOperator{\imply}{\Longrightarrow\qquad}
\DeclareMathOperator{\equi}{\Longleftrightarrow\qquad}
\DeclareMathOperator{\equivalent}{\quad\Longleftrightarrow\quad}
\DeclareMathOperator{\holds}{\text{ holds: }}
\DeclareMathOperator{\suchthat}{\text{ such that }}
\DeclareMathOperator{\into}{\text{ in }}
\DeclareMathOperator{\with}{\text{ with }}
\DeclareMathOperator{\aand}{\text{ and }}
\DeclareMathOperator{\natur}{\mathbb{N}}
\DeclareMathOperator{\real}{\mathbb{R}}
\DeclareMathOperator{\rationa}{\mathbb{Q}}
\DeclareMathOperator{\whole}{\mathbb{Z}}
\DeclareMathOperator{\divv}{\,\text{DIV}\,}
\DeclareMathOperator{\modd}{\,\text{MOD}\,}
\DeclareMathOperator{\argmax}{\text{argmax}}
\DeclareMathOperator{\argmin}{\text{argmin}}
\DeclareMathOperator{\sign}{\text{sign}}


% presentation metadata
\title{Clifford Tableaus and \\ the Stabilizer Algorithm}
\subtitle{}
\author{Leonard Uscinowicz}

\institute{\theChairName\\\theDepartmentName\\\theUniversityName}
\date[20/12/2024]{December 20\textsuperscript{th}, 2024}

\footline{\insertauthor~|~\insertshorttitle~|~\insertshortdate}

% macro to configure the style of the presentation
\TUMbeamersetup{
    title page = TUM tower,
    part page = TUM toc,
    section page = TUM toc,
    content page = TUM more space,
    tower scale = 1.0,
    headline = TUM threeliner,
    footline = TUM default,
    headline on = {title page},
    footline on = {every page, title page=false},
}

\usepackage{biblatex}
\addbibresource{bibliography.bib}

\begin{document}
    \renewcommand{\theChairName}{\(\qquad\)}
    \renewcommand{\theDepartmentName}{\(\qquad\)}
    \maketitle


    \section{Preliminary Definitions}
    \label{sec:preliminary-definitions}
    \begin{frame}{Pauli Matrixes}
        \[
            I=\begin{pmatrix}
                  1 & 0 \\ 0 & 1
            \end{pmatrix}
            \qquad
            X=\begin{pmatrix}
                  0 & 1 \\ 1 & 0
            \end{pmatrix}
            \qquad
            Y=\begin{pmatrix}
                  0 & -i \\ i & 0
            \end{pmatrix}
            \qquad
            Z=\begin{pmatrix}
                  1 & 0 \\ 0 & -1
            \end{pmatrix}
        \]
        \newline
        \textbf{Products of Pauli matrices:}
        \begin{gather*}
            I^2=X^2=Y^2=Z^2=I \\
            IX=XI=X
            \qquad
            IY=YI=Y
            \qquad
            IZ=ZI=Z \\
            \begin{aligned}
                &XY=iZ
                \qquad
                &&YX=-iZ \\
                &YZ=iX
                \qquad
                &&ZY=-iX \\
                &ZX=iY
                \qquad
                &&XZ=-iY
            \end{aligned}
        \end{gather*}
        \vspace*{4mm}

        {\footnotesize
        \cite*{02_ImprovedSimulationOfStabilizerCircuits}
        \fullcite{02_ImprovedSimulationOfStabilizerCircuits}
        }
    \end{frame}

    \begin{frame}{Group Theory}
        \textbf{Group} \((G,\cdot)\) is a non-empty set \(G\) \\
        with a binary group multiplication operation \("\cdot"\) \\
        with the properties: \newline
        \begin{itemize}
            \setlength{\itemsep}{1.25\baselineskip}
            \item \textbf{Closure:}
            \(\forall g_1,g_2\in G\Longrightarrow g_1\cdot g_2\in G\)
            \item \textbf{Associativity:}
            \(\forall g_1,g_2,g_3\in G\Longrightarrow g_1\cdot(g_2\cdot g_3)=(g_1\cdot g_2)\cdot g_3\)
            \item \textbf{Identity:}
            \(\exists e\in G\suchthat\forall g\in G\Longrightarrow e\cdot g=g\cdot e=g\)
            \item \textbf{Inverse:}
            \(\forall g\in G\Longrightarrow\exists g^{-1}\in G\suchthat g\cdot g^{-1}=g^{-1}\cdot g=e\)
        \end{itemize}
        \vspace*{4mm}

        {\footnotesize
        \cite*{01_QuantumComputationAndQuantumInformation}
        \fullcite{01_QuantumComputationAndQuantumInformation}
        }
    \end{frame}

    \begin{frame}{Pauli Group}
        \textbf{Pauli Group}
        \(\mathcal{P}_n\)
        is defined as the group of \(n\)-qubit Pauli operators. \\
        It consists of all tensor products of \(n\) Pauli matrices, \\
        together with a phase factor of \(\pm 1\) or \(\pm i\).
        \begin{gather*}
            \mathcal{P}_1=\left\{
            \pm I,\pm iI,
            \pm X,\pm iX,
            \pm Y,\pm iY,
            \pm Z,\pm iZ
            \right\} \\
            \mathcal{P}_n
            =\left\{
            \left.
            i^m\bigotimes_{j=1}^n\sigma_{k_j}
            \right|
            m,k_j\in\left\{0,1,2,3\right\},
            \sigma_0=I,
            \sigma_1=X,
            \sigma_2=Y,
            \sigma_3=Z
            \right\}
        \end{gather*}
        \newline
        Size of a Pauli Group:
        \(\left|\mathcal{P}_n\right|=4^{n+1}\)
        \vspace*{10mm}

        {\footnotesize
        \cite*{02_ImprovedSimulationOfStabilizerCircuits}
        \fullcite{02_ImprovedSimulationOfStabilizerCircuits} \\
        \cite*{01_QuantumComputationAndQuantumInformation}
        \fullcite{01_QuantumComputationAndQuantumInformation}
        }
    \end{frame}

    \begin{frame}{Pauli Group Operation}
        Given two Pauli operators
        \(P=i^{m_P}\bigotimes_{j=1}^{n}P_j\)
        and
        \(Q=i^{m_Q}\bigotimes_{j=1}^{n}Q_j\),
        their product, as necessitated by Group Definition, is:
        \[
            P\cdot Q=i^{m_P+m_Q}\bigotimes_{j=1}^{n}P_jQ_j
        \]
        \newline
        \(P\) commutes with \(Q\) if the number of indices \(j\) such that \(P_j\) anti-commutes with \(Q_j\) is even.
        \vspace*{25mm}

        {\footnotesize
        \cite*{02_ImprovedSimulationOfStabilizerCircuits}
        \fullcite{02_ImprovedSimulationOfStabilizerCircuits}
        }
    \end{frame}

    \begin{frame}{Group Generators}
        A set of \(l\) elements \(\left\{g_i\right\}_{1\leq i\leq l}\)
        generates a group \(G\) if every element \(g\in G\) can be written as a product of the generators. \\
        In this case, the group \(G\) can be written in terms of its generators:
        \[
            G=\left\langle g_i\mid i\in\natur,1\leq i\leq l\right\rangle
        \]
        \[
            \textbf{Examples:}\qquad
            \begin{matrix}
                \mathcal{P}_1=\left\langle X,Z,iI\right\rangle \\
                \left\langle X\right\rangle = \left\{I,X\right\}
            \end{matrix}
        \]
        \vspace*{20mm}

        {\footnotesize
        \cite*{01_QuantumComputationAndQuantumInformation}
        \fullcite{01_QuantumComputationAndQuantumInformation}
        }
    \end{frame}


    \section{Stabilizer Formalism}
    \label{sec:stabilizer-formalism}

    \begin{frame}{Stabilizer Groups}{Definitions}
        \begin{itemize}
            \setlength{\itemsep}{1.25\baselineskip}
            \item
            Element \(g\in \mathcal{P}_n\) \textbf{stabilizes} \(\ket{\psi}\) iff \(g\ket{\psi}=\ket{\psi}\). \\
            \(\ket{\psi}\) is eigenstate of \(g\) with eigenvalue \(+1\).
            \item
            \(S\widehat{=}\)Subgroup of the Pauli Group \(\mathcal{P}_n\): \(S\subseteq\mathcal{P}_n\).
            \item
            \(V_S\widehat{=}\)Set of \(n\)-qubit states stabilized by \(S\):
        \end{itemize}
        \vspace*{4mm}
        \[
            V_S=\left\{
            \ket{\psi}
            \mid
            S\subseteq \mathcal{P}_n,
            \forall g\in S\holds
            g\ket{\psi}=\ket{\psi}
            \right\}
        \]
        \vspace*{10mm}

        {\footnotesize
        \cite*{01_QuantumComputationAndQuantumInformation}
        \fullcite{01_QuantumComputationAndQuantumInformation}
        }
    \end{frame}


    \begin{frame}{Stabilizer Groups}{Properties}
        Not just any subgroup \(S\) of the Pauli group can be used as the stabilizer
        for a non-trivial vector space \(V_S\). \\
        \vspace*{2mm}
        \textbf{Example:}
        \(S=\{\pm I,\pm X\}\)
        \[
            (-I)\in S
            \aand
            (-I)\ket{\psi}=-\ket{\psi}
            \means
            \ket{\psi}=\Vec{0}
            \means
            V_S=\begin{Bmatrix}
                    \Vec{0}
            \end{Bmatrix}\,\text{(trivial)}
        \]

        \vspace*{3mm}
        Conditions for \(S\) such that \(V_S\) not trivial:
        \vspace*{2mm}
        \begin{itemize}
            \setlength{\itemsep}{0.75\baselineskip}
            \item \textbf{Commutativity:}
            \(\forall g_1,g_2\in S\holds g_1g_2=g_2g_1\)
            \item \textbf{Strict Identity:}
            \(-I\not\in S,\,iI\not\in S,\,-iI\not\in S,\)
        \end{enumerate}
        \vspace*{7mm}

        {\footnotesize
        \cite*{01_QuantumComputationAndQuantumInformation}
        \fullcite{01_QuantumComputationAndQuantumInformation}
        }
    \end{frame}

    \begin{frame}{Stabilizer Conditions}{Commutativity Proof}
        \vspace*{-3mm}
        Let \(V_S\) be non-trivial. \\

        \vspace*{2mm}
        Let \(g_1,g_2\in S\). \\

        \vspace*{2mm}
        \(\Longrightarrow\)
        \(g_1\) and \(g_2\) are tensor products of Pauli matrices. \\

        \vspace*{2mm}
        \(\Longrightarrow\)
        \(g_1\) and \(g_2\) must either commute or anti-commute. \\

        \vspace*{2mm}
        \(\qquad\)
        Suppose \(g_1\) and \(g_2\) anti-commute:
        \[
            \ket{\psi}
            =g_1g_2\ket{\psi}
            =-g_2g_1\ket{\psi}
            =-\ket{\psi}
            \equivalent
            \ket{\psi}=\Vec{0}
            \means
            V_S\text{ is trivial.}
        \]

        \vspace*{2mm}
        \(\Longrightarrow\)
        \(g_1\) and \(g_2\) anti-commuting leads to a contradiction. \\

        \vspace*{2mm}
        \(\Longrightarrow\)
        \(g_1\) and \(g_2\) commute.

        \vspace*{5mm}
        {\footnotesize
        \cite*{01_QuantumComputationAndQuantumInformation}
        \fullcite{01_QuantumComputationAndQuantumInformation}
        }
    \end{frame}



















    \begin{frame}{Refernces}
        \printbibliography
    \end{frame}

\end{document}
